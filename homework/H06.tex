

22. 用正交线性变换将二次型 $f=x_1^2+2x_2^2+3x_3^2+4x_1x_2-4x_2x_3$ 化为标准型, 并给出所用的正交线性变换.

\textbf{Sol}: 系数矩阵 $A=\begin{bmatrix}
    1&2&0\\
    2&2&-2\\
    0&-2&3\\
\end{bmatrix}$, 则 $A-\lambda I=\begin{bmatrix}
    1-\lambda&2&0\\
    2&2-\lambda&-2\\
    0&-2&3-\lambda
\end{bmatrix}=(\lambda-5)(\lambda+1)(\lambda-2)=0$. 得到 $A$ 得特征值为 $\lambda_1=-1,\lambda_2=2,\lambda_3=5$.

对于 $\lambda_1=-1$, 解线性方程组 $(A-\lambda I)x=0$, 得一个基础解系 $x_1=[2,-2,-1]^T$, 单位化后得 $\varepsilon_1=\frac{1}{3}[2,-2,-1]^T$.

对于 $\lambda_2=2$, 解线性方程组 $(A-\lambda I)x=0$, 得一个基础解系 $x_2=[2,1,2]^T$, 单位化后得 $\epsilon_2=\frac{1}{3}[2,1,2]^T$.

对于 $\lambda_3=5$, 解线性方程组 $(A-\lambda I)x=0$, 得一个基础解系 $x_3=[1,2,-2]^T$, 单位化后得 $\varepsilon_3=\frac{1}{3}[1,2,-2]^T$.

于是, 令 $Q=[\varepsilon_1\;\varepsilon_2\;\varepsilon_3]
=\mymathbox{
\dfrac{1}{3}\begin{bmatrix}
    2&2&1\\
    -2&1&2\\
    -1&2&-2\\
\end{bmatrix}^T}$
, 则 $Q$ 为正交矩阵, 
且经正交线性变换 $x=Qy$, 二次型 $f$ 转化为 $\mymathbox{f=-y_1^2+2y_2^2+5y_3^2}$.

\vspace{12pt}

23. 问 $t$ 取何值时, 对称矩阵 $A$ 是正定的, 其中 

$$
A=
\begin{bmatrix}
    1&t&1\\
    t&2&0\\
    1&0&1-t\\
\end{bmatrix}
$$

\textbf{Sol}: $f$ 为正定二次型得充要条件是, 顺序主子式都大于 0, 即 $1>0$, $\begin{vmatrix}1&t\\t&2\\\end{vmatrix}=2-t^2>0$, $\begin{vmatrix}
    1&t&1\\
    t&2&0\\
    1&0&1-t
\end{vmatrix}=t(t+1)(t+2)>0$. 

解得 \mymathbox{t\in(-1,0)}, 此时 $f$ 为正定二次型.


\vspace{12pt}

24. 随机地从一批钉子中抽取 16 只, 测得其长度(单位:厘米)为
$$
\begin{aligned}
&2.14,2.10,2.13,2.15,2.13,2.12,2.13,2.10,\\
&2.15,2.12,2.14,2.10,2.13,2.11,2.14,2.11\\
\end{aligned}
$$
假定钉长分布是正态的, 求总体均值 $\mu$ 的双侧 $90\%$ 置信区间:
\begin{enumerate}[(1)]
    \item 若已知 $\sigma=0.01$ 厘米;
    \item 若 $\sigma$ 未知.
\end{enumerate}

\textbf{Sol}: 

\begin{enumerate}[(1)]
    \item 一个正态总体下 $\sigma^2$ 已知, 
    随机变量 $J=\sqrt{n}\dfrac{(\bar{X}-\mu)}{\sigma}$ 服从标准正态分布, 
    因此取 $\alpha=0.1$. 则 $\mu\in[\mu_0,\mu_1]$, 
    $\mu_0=\bar{X}-\mu_{1-\frac{\alpha}{2}}\dfrac{\sigma}{\sqrt{n}}$, 
    $\mu_1=\bar{X}+\mu_{1-\frac{\alpha}{2}}\dfrac{\sigma}{\sqrt{n}}$. 得到 $\bar{X}=2.125$, 即 $\mu_0=2.125-1.6448\cdot\dfrac{0.01}{\sqrt{16}}=2.1209$, 同理 $\mu_1=2.1291$. 
    \mymathbox{[2.1209,2.1291]}
    \item 一个正态总体下 $\sigma^2$ 未知, 随机变量 $J=\sqrt{n}\dfrac{(\bar{X}-\mu)}{S}$ 服从 $t(n-1)$ 分布, 取 $\alpha=0.01$, 则 $\mu\in[\mu_0,\mu_1]$, $\mu_0=\bar{X}-t_{1-\frac{\alpha}{2}}(n-1)\dfrac{S}{\sqrt{n}}$, $\mu_1=\bar{X}+t_{1-\frac{\alpha}{2}}(n-1)\dfrac{S}{\sqrt{n}}$. 得到 $\bar{X}=2.125,\;S=0.0171$, 即 $\mu_0=2.125-1.7531\cdot\dfrac{0.0171}{\sqrt{16}}=2.1175$, $\mu_1=2.1325$. 
    \mymathbox{[2.1175,2.1325]}
\end{enumerate}


\vspace{12pt}

25. 甲乙两位化验员独立地对一种聚合物的含氯量用相同的方法各做了 10 次测定, 得 $s^2=0.5419,\;s_2^2=0.6050$. 求他们测定值得方差比得双侧 $90\%$ 置信区间, 假定测定值服从正态分布.

\textbf{Sol}: 两个正态总体下 $\mu_1,\mu_2$ 未知, 随机变量 $J=\dfrac{s_1^2/\sigma_1^2}{s_2^2/\sigma_2^2}$ 服从 $F(m-1,n-1)$ 分布, 因此, 取 $\alpha=0.1$, 则 $\dfrac{\sigma_1^2}{\sigma_2^2}\in[k_0,k_1]$, $k_0=\dfrac{s_1^2}{s_2^2}\cdot\dfrac{1}{F_{1-\frac{\alpha}{2}}(m-1,n-1)}$, $k_1=\dfrac{s_1^2}{s_2^2}\cdot\dfrac{1}{F_{\frac{\alpha}{2}}(m-1,n-1)}$.

得到, $k_0=\dfrac{s_1^2}{s_2^2}\cdot\dfrac{1}{F_{1-\frac{\alpha}{2}}(m-1,n-1)}=\dfrac{0.5419}{0.6050}\cdot\dfrac{1}{F_{0.95}(9,9)}=0.2818$, $k_1=\dfrac{s_1^2}{s_2^2}\cdot\dfrac{1}{F_{\frac{\alpha}{2}}(m-1,n-1)}=\dfrac{0.5419}{0.6050}\cdot\dfrac{1}{F_{0.05}(9,9)}=2.8471$.

\mymathbox{[0.2818,2.8471]}

\vspace{12pt}

26. 设从某种型号的一大批晶体管中随机抽取 100 只样品, 测得其寿命标准差 $s=45$ 小时, 求这批晶体管寿命标准差 $\sigma$ 得双侧 $95\%$ 置信区间.

\textbf{Sol}: 根据辛钦中心极限定理, 设随机变量 $x_1,x_2,\cdots,x_n$ 相互独立, 服从同一分布且有有限的数学期望 $\mu$ 和方差 $\sigma^2$, 
则随机变量 $\bar{x}=\dfrac{\sum x_i}{n}$, 满足 $\bar{x}\sim N(\mu,\dfrac{\sigma^2}{n})$. 一个正态总体在 $\mu$ 未知时, 
$J=\dfrac{1}{\sigma^2}\sum_{i=1}^n(X_i-\bar{X})^2$ 服从 $\chi^2(n-1)$ 分布, 则有 $\sigma^2\in[\sigma_0^2,\sigma_1^2]$. $\sigma_0^2=\dfrac{\sum_{i=1}^n(X_i-\bar{X})^2}{\chi_{1-\frac{\alpha}{2}}^2(n-1)}$, $\sigma_1^2=\dfrac{\sum_{i=1}^n(X_i-\bar{X})^2}{\chi_{\frac{\alpha}{2}}^2(n-1)}$, 得到 $\sigma_0^2=\dfrac{\sum_{i=1}^n(X_i-\bar{X})^2}{\chi_{1-\frac{\alpha}{2}}^2(n-1)}=\dfrac{(n-1)S^2}{\chi_{1-\frac{\alpha}{2}}^2(n-1)}=\dfrac{99\cdot45^2}{\chi^2_{0.975}(99)}\approx1561.1$, 同理可得 $\sigma_1^2\approx2732.7$, 即 $\sigma^2\in[1561.1, 2732.7]$. 
因此标准差的 $90\%$ 置信区间为:  \mymathbox{\sigma\in[39.51, 52.28]}.


\vspace{12pt}


