
16. 已知 3 阶方阵 $A$ 的属于特征值 $1,0,-1$ 的特征向量依次为 $x_1=[1,2,3]^T$, 
$x_2=[2,-2,1]^T$, $x_3=[-2,-1,2]^T$, 求 $A$ 和 $A^8$.

\textbf{Sol}: 令 $X=[x_1 \mid x_2 \mid x_3]$, $AX=X\begin{bmatrix}
    1&&\\
    &0&\\
    &&-1
\end{bmatrix}$,

因 $|X|=-3\not=0$, 有 $A=X\begin{bmatrix}
    1&&\\
    &0&\\
    &&-1\\
\end{bmatrix}X^{-1}$, 则 $X^{-1}=\dfrac{1}{9}\begin{bmatrix}
    1&2&2\\
    2&-2&1\\
    -2&-1&2\\
\end{bmatrix}$.

$A=X\begin{bmatrix}
    1&&\\
    &0&\\
    &&-1\\
\end{bmatrix}X^{-1}
=\mybox{
$\dfrac{1}{3}\begin{bmatrix}
    -1&0&2\\
    0&1&2\\
    2&2&0\\
\end{bmatrix}$
}$.

$A^8=X\begin{bmatrix}
    1&&\\
    &0&\\
    &&-1\\
\end{bmatrix}^8X^{-1}=X\begin{bmatrix}
    1&&\\
    &0&\\
    &&1\\
\end{bmatrix}X^{-1}
=\mybox{
$\dfrac{1}{9}\begin{bmatrix}
    5&4&-2\\
    4&5&2\\
    -2&2&8
\end{bmatrix}$
}$.


\vspace{12pt}

17. 求正交矩阵 $Q$ 使 $Q^{-1}AQ$ 为对角矩阵, 其中

$$A=
\begin{bmatrix}
5  & -4 & 2\\
-4 & 5  & -2\\
2  & -2 & 2\\
\end{bmatrix}.
$$

\textbf{Sol}:
求 $A$ 的特征值有 $|A-\lambda I|=0$, 得
$$
|A-\lambda I|=
\begin{vmatrix}
5-\lambda &-4&2\\
-4&5-\lambda &-2\\
2&-2&2-\lambda 
\end{vmatrix}=(\lambda-1)^2(10-\lambda)=0
$$

因此得到 $A$ 的特征值为 $\lambda_1=10,\lambda_2=\lambda_3=1$. 对 $\lambda_1=10$, 由特征方程 $|A-\lambda_1 I|X=0$ 得到一个解 $X_1=[-2,2,-1]^T$, 单位化得到 $\varepsilon_1=[-\frac{2}{3},\frac{2}{3},-\frac{1}{3}]^T$.

对于 $\lambda_2=\lambda_3=1$, 由 $|A-\lambda_2I|X=0$ 得到 $2x_1-2x_2+x_3=0$ 及一个解 $x_2=[1,1,0]^T$, $x_3=[-1,1,4]^T$, 单位化得到 $\varepsilon_2=[\frac{1}{\sqrt{2}},\frac{1}{\sqrt{2}},0]^T$, $\varepsilon_3=[-\frac{1}{3\sqrt{2}},\frac{1}{3\sqrt{2}},\frac{4}{3\sqrt{2}}]^T$.

令 $\mybox{
$Q=[\varepsilon_1 \; \varepsilon_2 \; \varepsilon_3]=\begin{bmatrix}
    -\frac{2}{3} & \frac{\sqrt{2}}{2} & -\frac{\sqrt{2}}{6}\\
    \frac{2}{3} & \frac{\sqrt{2}}{2} & \frac{\sqrt{2}}{6}\\
    -\frac{1}{3} & 0 & \frac{2\sqrt{2}}{3}
\end{bmatrix}$
}$, 
$Q^{-1}AQ = \Lambda = 
\begin{bmatrix}
    10&&\\
    &1&\\
    &&1
\end{bmatrix}$.

\vspace{12pt}

18. 设 $A=\begin{bmatrix}
    -1&2&4\\
    2&x&2\\
    4&2&-1\\
\end{bmatrix}$ 与 $B=\begin{bmatrix}
    5&&\\&y&\\&&-5
\end{bmatrix}$ 相似, 求 $x,y$.

\textbf{Sol}: 由于 $A\sim B$, 因此有 $|A|=|B|$ 且 $tr(A) = tr(B)$.

$$
\begin{cases}
    x+16+16-16x+4+4=-25y\\
    -1+x-1=5+y-5
\end{cases}
\Rightarrow
\mybox{
$\begin{cases}
    x = 1\\
    y = -1
\end{cases}$
}
$$

\vspace{12pt}

19. 设样本 $(1.3,\;0.6,\;1.7,\;2.2\;0.3,\;1.1)$ 是取自具有概率密度

$$
f(x,\beta)=\begin{cases}
\dfrac{1}{\beta},&0<x<\beta\\
0,&\text{else}.
\end{cases}
$$
的总体, 用矩估计法估计总体均值、总体方差及参数 $\beta$.

\textbf{Sol}:
总体均值: $\alpha_1=E(x)$, 用矩估计法估计得到\mybox{ $\hat{\alpha}_1=\bar{x}=1.2$}.\par
总体方差: $\alpha_2=M_k'(x)$, 用矩估计法得到 \mybox{$\hat{\alpha}_2=S_x=0.488$}.\par
利用总体均值估计参数: 由于

$$
\alpha_1=E(x)=\int_{-\infty}^{+\infty}xf(x,\beta)\text{d}x=\int_0^{\beta}\dfrac{x}{\beta}\text{d}x=\dfrac{x^2}{2\beta}\Bigg|_{0}^{\beta}=\dfrac{\beta}{2}
$$
则有 $\beta=2\alpha_1$。所以 $\beta$ 矩估计值为 \mybox{$\hat{\beta}_1=2\bar{x}=2.4$}.

\vspace{12pt}

20. 设总体 $X$ 服从对数正态分布 $LN(\mu,\sigma^2)$, 其概率函数为

$$
f(x,\mu,\sigma^2)=\begin{cases}
    \dfrac{1}{\sqrt{2\pi}\sigma x}\exp\Bigg[-\dfrac{1}{2\sigma^2}(\ln x-\mu)^2\Bigg],&x>0,\\
    0,&x\leqslant0.
\end{cases}
$$
其中, $\mu,\sigma^2$ 均未知, $X_1,X_2,\cdots,X_n$ 是取自这个总体的样本, 求 $\mu$ 与 $\sigma^2$ 的极大似然估计量.

\textbf{Sol}: 似然函数为

$$
\begin{aligned}
    L(\mu,\sigma^2)
=&\prod_{i=1}^nf(x_i,\mu,\sigma^2)\\
=&\Big(\dfrac{1}{\sqrt{2\pi}\sigma}\Big)^n
\prod_{i=1}^nx_i^{-1}
\exp \Big(-\dfrac{1}{2\sigma^2}\sum_{i=1}^n(\ln x_i-\mu)^2\Big)
\end{aligned}
$$

$$
\begin{aligned}
    &\ln L(\mu,\sigma^2)\\
    =&-\dfrac{n}{2}\ln2\pi-\dfrac{n}{2}\ln\sigma^2-\sum_{i=1}^nx_i-\dfrac{1}{2\sigma^2}\Big(\sum_{i=1}^n(\ln x_i-\mu)^2\Big)
\end{aligned}
$$

故似然方程组为
$$
\begin{cases}
\displaystyle
    \dfrac{\partial}{\partial \mu }\ln L=\dfrac{1}{2\sigma^2}\sum_{i=1}^n(2\ln x_i-2\mu)=0\\
\displaystyle
    \dfrac{\partial}{\partial \sigma^2}\ln L=-\dfrac{n}{2\sigma^2}+\dfrac{1}{2(\sigma^2)^2}\Big(\sum_{i=1}^n(\ln x_i-\mu)^2\Big)=0
\end{cases}
$$

解得 \mybox{$\begin{cases}
\displaystyle
    \mu=\dfrac{1}{n}\sum_{i=1}^n\ln x_i\\
\displaystyle
    \sigma^2=\dfrac{1}{n}\sum_{i=1}^n\Big(\ln x_i-\dfrac{1}{n}\sum_{i=1}^n\ln x_i\Big)^2
\end{cases}$}

\vspace{12pt}

21. 设 $X_1,X_2,X_3$ 是取自总体 $X$ 的样本, 证明下列统计量都是总体均值 $E(X)$ 的无偏估计量:

$$
\begin{aligned}
    &t_1(X_1,X_2,X_3)=\dfrac{2}{5}X_1+\dfrac{1}{5}X_2+\dfrac{2}{5}X_3;\\
    &t_2(X_1,X_2,X_3)=\dfrac{1}{6}X_1+\dfrac{1}{3}X_2+\dfrac{1}{2}X_3;\\
    &t_3(X_1,X_2,X_3)=\dfrac{1}{7}X_1+\dfrac{3}{14}X_2+\dfrac{9}{14}X_3.
\end{aligned}
$$

并问哪一个无偏估计量方差最小?

\textbf{Sol}: 上述式子具有一般形式 $t_i(X_1,X_2,X_3)=w_{1,i}X_1+w_{2,i}X_2+w_{3,i}X_3$, $\sum_{j=1}^3w_{j,i}=1$,  

$E[t_i(X_1,X_2,X_3)-E(X)]=\dfrac{1}{3}\sum_{j=1}^3w_{j,i}E[X_j]-E(X)\Big(\sum_{j=1}^3w_{j,i}-1\Big)=0$,

\mybox{从而得证上述式子为 $E(x)$ 的无偏估计量}. 其均方误差

$$
\begin{aligned}
    r
    =&E[t_i(X_1,X_2,X_3)-E(x)]^2\\
    =&E[t_i(X_1,X_2,X_3)^2]+E[E(X)]^2-2E[t_i(X_1,X_2,X_3)E(X)]\\
    =&\dfrac{1}{3}\sum_{j=1}^3w_{j,i}^2X_{j}^2+\dfrac{3}{3}E(X)^2-\dfrac{2}{3}E(X)\sum_{j=1}^3w{j,i}X_j\\
    =&\sum_{j=1}^3[w_{j,i}^2E[X_j-E(X)]^2]\\
    =&\sum_{j=1}^3w_{j,i}^2\cdot S(X)
\end{aligned}
$$


均方误差 $r(t_1)=\dfrac{9}{25}S(X)$, $r(t_2)=\dfrac{7}{18}S(X)$, $r(t_3)=\dfrac{94}{196}S(X)$, 因此\mybox{无偏估计量 $t_1$ 方差最小}.
\vspace{12pt}

个人感觉更好的答案:\\

$\begin{aligned}
    &E(t_1)=E(\dfrac{2}{5}X_1+\dfrac{1}{5}X_2+\dfrac{2}{5}X_3)=\dfrac{2}{5}E(X_1)+\dfrac{1}{5}E(X_2)+\dfrac{2}{5}E(X_3)=E(X)\\
    &E(t_2)=E(\dfrac{1}{6}X_1+\dfrac{1}{3}X_2+\dfrac{1}{2}X_3)=\dfrac{1}{6}E(X_1)+\dfrac{1}{3}E(X_2)+\dfrac{1}{2}E(X_3)=E(X)\\
    &E(t_3)=E(\dfrac{1}{7}X_1+\dfrac{3}{14}X_2+\dfrac{9}{14}X_3)=\dfrac{2}{5}E(X_1)+\dfrac{1}{5}E(X_2)+\dfrac{2}{5}E(X_3)=E(X)\\
\end{aligned}$

因此上述式子为 $E(X)$ 的无偏估计量

$\begin{aligned}
    &D(t_1)=D(\dfrac{2}{5}X_1+\dfrac{1}{5}X_2+\dfrac{2}{5}X_3)=(\dfrac{4}{25}+\dfrac{1}{25}+\dfrac{4}{25})D(X)=\dfrac{9}{25}D(X)\\
    &D(t_2)=D(\dfrac{1}{6}X_1+\dfrac{1}{3}X_2+\dfrac{1}{2}X_3)=(\dfrac{1}{36}+\dfrac{1}{9}+\dfrac{1}{4})D(X)=\dfrac{7}{18}D(X)\\
    &D(t_3)=D(\dfrac{1}{7}X_1+\dfrac{3}{14}X_2+\dfrac{9}{14}X_3)=(\dfrac{1}{49}+\dfrac{9}{196}+\dfrac{81}{196})=\dfrac{47}{98}D(X)\\
\end{aligned}$

因此可知无偏估计量 $t_1$ 的方差最小.