

43. 利用 $\text{Gerschgorin}$ 定理确定方阵

$$
\textbf{A}=\begin{bmatrix}
    1&-1&0\\
    1&5&1\\
    -2&-1&9\\
\end{bmatrix}
$$

的特征值范围. 判断 $\textbf{A}$ 的特征值是否都是实数?

\textbf{Sol}:  

$\textbf{A}$ 的三个盖氏圆分别为

$$
\begin{aligned}
    &G_1=\{z|\mid z-1\mid \leqslant 1\}\\
    &G_2=\{z|\mid z-5\mid \leqslant 2\}\\
    &G_1=\{z|\mid z-9\mid \leqslant 3\}\\
\end{aligned}
$$

$G_2,\;G_3$ 相交, 因此 $\lambda_1\in G_1,\;\lambda_2,\lambda_3\in G_2\bigcup G_3$.

对矩阵 $\textbf{A}^{\text{T}}$ 应用圆盘定理. 矩阵 $\textbf{A}^{\text{T}}$ 对应的三个圆盘为

$$
\begin{aligned}
    &G_1'=\{z|\mid z-1\mid \leqslant 3\}\\
    &G_2'=\{z|\mid z-5\mid \leqslant 2\}\\
    &G_1'=\{z|\mid z-9\mid \leqslant 1\}\\
\end{aligned}
$$

综合分析可知三个特征值属于 $\lambda_1\in G_1,\;\lambda_2\in G_2,\;\lambda_3\in G_3'$, 
发现 $G_1,\;G_2,\;G_3$ 相互独立, 因此可知 $\lambda_1,\;\lambda_2,\;\lambda_3$ \mytextbox{都为实数}, 且特征值的范围为

\mymathbox{0\leqslant\lambda_1\leqslant2$, $3\leqslant\lambda_2\leqslant 7$, $8\leqslant \lambda_3 \leqslant 10}.

\vspace{12pt}


44. 已知方程 $x^3-x^2-1=0$ 在 $x^{(0)}=1.5$ 附近有根, 将方程改写成:

\begin{enumerate}[(1)]
    \item $x=1+\dfrac{1}{x^2}$, 对应的迭代算式为 $x^{(k+1)}=1+\dfrac{1}{[x^{(k)}]^2}$;
    \item $x=\sqrt[3]{1+x^2}$, 对应的迭代算式为 $x^{(k+1)}=\sqrt[3]{1+[x^{(k)}]^2}$;
    \item $x=\dfrac{1}{\sqrt{x-1}}$, 对应的迭代算式为 $x^{(k+1)}=\dfrac{1}{\sqrt{x^{(k)}}-1}$.
\end{enumerate}

判断上述各种迭代算式在 $1.5$ 附近的收敛性.

\textbf{Sol}:

\begin{enumerate}[(1)]
    \item $\varphi(x)=1+x^{-2}$, $\varphi'(x)=(1+x^2)=-2x^{-3}$, $|\varphi'(1.5)|=|-2\times 1.5^{-3}|\approx 1.78>1$, 该算式在 1.5 附近\mytextbox{不收敛}.
    \item $\varphi(x)=(1+x^2)^{\frac{1}{3}}$, $\varphi'(x)=\dfrac{2x}{3}(1+x^2)^{-\frac{2}{3}}$, $|\varphi'(1.5)|\approx 0.456 <1$, 该算式在 1.5 附近\mytextbox{收敛}.
    \item $\varphi(x)=(x-1)^{-\frac{1}{2}}$, $\varphi'(x)=-\dfrac{1}{2}(x-1)^{-\frac{3}{2}}$, $|\varphi'(1.5)|\approx 1.414 > 1$, 该算式在 1.5 附近\mytextbox{收敛}.
\end{enumerate}

\vspace{12pt}


