

37.求下列方阵 $A$ 的函数 $e^A$ 和 $e^{At}$:

$$
A=\begin{bmatrix}
    2&1&3\\
    0&-1&0\\
    2&3&1\\
\end{bmatrix}
$$


\textbf{Sol}:  $|A-\lambda I|=-(\lambda+1)^2(\lambda-4)=0$, 得特征值 $\lambda_!=\lambda_2=-1,\lambda_3=4$, 对应的 $\lambda_1,\;\lambda_2$ 得特征向量为 $[1,\;0,\;-1]^{\text{T}},\;[\frac{4}{7},\;-\frac{5}{7},\;0]^{\text{T}}$, 对应的 $\lambda_3$ 的特征向量为 $[3,\;0,\;2]^{\text{T}}$.

则 $P=\begin{bmatrix}
    3&1&\dfrac{4}{7}\\
    0&0&-\dfrac{5}{7}\\
    2&-1&0\\
\end{bmatrix},\;P^{-1}=-\dfrac{1}{25}\begin{bmatrix}
    5&4&5\\
    10&8&-15\\
    0&-35&0\\
\end{bmatrix}$. 于是, 有 $A=P\begin{bmatrix}
    4&&\\
    &-1&1\\
    &&-1\\
\end{bmatrix}P^{-1}$, 因此

$$
\mybox{
$\begin{aligned}
    e^{\text{A}}
    =&\dfrac{1}{25}\begin{bmatrix}
        3&1&\dfrac{4}{7}\\
        0&0&-\dfrac{5}{7}\\
        2&-1&0\\
    \end{bmatrix}
    \begin{bmatrix}
        e^4&&\\
        &e^{-1}&e^{-1}\\
        &&e^{-1}\\
    \end{bmatrix}
    \begin{bmatrix}
        5&4&5\\
        10&8&-15\\
        0&-35&0\\
    \end{bmatrix}\\
    =&\dfrac{1}{25}\begin{bmatrix}
        15e^{4}+10e^{-1}&12e^4-47e^{-1}&15e^{4}-15e^{-1}\\
        0&25e^{-1}&0\\
        10e^{4}-10e^{-1}&8e^{4}+27e^{-1}&10e^{4}+15e^{-1}\\
    \end{bmatrix}
\end{aligned}$
}
$$

$$
\mybox{
$\begin{aligned}
    e^{\text{A}t}
    =&\dfrac{1}{25}\begin{bmatrix}
        3&1&\dfrac{4}{7}\\
        0&0&-\dfrac{5}{7}\\
        2&-1&0\\
    \end{bmatrix}
    \begin{bmatrix}
        e^{4t}&&\\
        &e^{-t}&te^{-t}\\
        &&e^{-t}\\
    \end{bmatrix}
    \begin{bmatrix}
        5&4&5\\
        10&8&-15\\
        0&-35&0\\
    \end{bmatrix}\\
    =&\dfrac{1}{25}\begin{bmatrix}
        15e^{4t}+10e^{-t}&12e^{4t}-(12+35t)e^{-t}&15e^{4t}-15e^{-t}\\
        0&25e^{-t}&0\\
        10e^{4t}-10e^{-t}&8e^{4t}-(8-35t)e^{-t}&10e^{4t}+15e^{-t}\\
    \end{bmatrix}
\end{aligned}$
}
$$

\vspace{12pt}

38. 证明: $A=\begin{bmatrix}
    4&-1&1\\
    -1&4.25&2.75\\
    1&2.75&3.5\\
\end{bmatrix}$ 是正定矩阵, 并求它的 $LDL^T$ 分解和 $\text{Choelsky}$ 分解. 

\textbf{Sol}:  因为 $\Delta_1=4>0,\;\Delta_2=16>0,\;\Delta_3=16>0$, 所以 $A$ 是正定矩阵. 根据 $A=LDL^{\text{T}}=\begin{bmatrix}
    1&&\\l_{21}&1&\\l_{31}&l_{32}&1\\
\end{bmatrix}
\begin{bmatrix}
    d_1&&\\&d_2&\\&&d_3\\
\end{bmatrix}
\begin{bmatrix}
    1&l_{21}&l_{31}\\&1&l_{32}\\&&1\\
\end{bmatrix}$, 可解得

$$
\begin{cases}
    d_1=4,\\
    l_{21}=-\dfrac{1}{4},\\
    l_{31}=\dfrac{1}{4},\\
    d_2=4,\\
    l_{32}=\dfrac{3}{4},\\d_3=1
\end{cases}
$$

即得到

$$
\begin{aligned}
    A=&LDL^{\text{T}}=LD^{\frac{1}{2}}D^{\frac{1}{2}}L^{\text{T}}=GG^{T}\\
    =&\mybox{
    $\begin{bmatrix}
        1&&\\-\dfrac{1}{4}&1&\\\dfrac{1}{4}&\dfrac{3}{4}&1
    \end{bmatrix}\begin{bmatrix}
        4&&\\&4&\\&&1
    \end{bmatrix}\begin{bmatrix}
        1&-\dfrac{1}{4}&\dfrac{1}{4}\\
        &1&\dfrac{3}{4}\\&&1
    \end{bmatrix}$
    }\\
    =&\begin{bmatrix}
        1&&\\-\dfrac{1}{4}&1&\\\dfrac{1}{4}&\dfrac{3}{4}&1
    \end{bmatrix}\begin{bmatrix}
        2&&\\&2&\\&&1
    \end{bmatrix}\begin{bmatrix}
        2&&\\&2&\\&&1
    \end{bmatrix}\begin{bmatrix}
        1&-\dfrac{1}{4}&\dfrac{1}{4}\\
        &1&\dfrac{3}{4}\\&&1
    \end{bmatrix}\\
    =&\mybox{
    $\begin{bmatrix}
        2&&\\-\dfrac{1}{2}&2&\\\dfrac{1}{2}&\dfrac{3}{2}&1\\
    \end{bmatrix}\begin{bmatrix}
        2\\-\dfrac{1}{2}&\dfrac{1}{2}\\
        &2&\dfrac{3}{2}\\
        &&1\\
    \end{bmatrix}$
    }
\end{aligned}
$$
\vspace{12pt}

39. 设

$$
A=\begin{bmatrix}
    1&1&2\\
    1&2&1\\
    1&1&3\\
    2&3&3\\
\end{bmatrix},\quad
b=\begin{bmatrix}
    1\\0\\2\\1,
\end{bmatrix}
$$

证明方程组 $Ax=b$ 有解, 并用 $\text{QR}$ 分解方法解方程组 $Ax=b$.

\textbf{Sol}:  因为 $r([A|b])=r(A)=3$, 因此方程组 $Ax=b$ 有解.

有 $A_1=[1\;1\;1\;2]^{\text{T}},\;A_2=[1\;2\;1\;3]^{\text{T}},\;A_3=[2\;1\;3\;3]^{\text{T}}$,

$\beta_1=A_1,\;\varepsilon_1=\dfrac{1}{\sqrt{7}}[1\;1\;1\;2]^{\text{T}}$,
$\beta_2=A_2-(A_2,\varepsilon_1)\varepsilon_1=\dfrac{1}{7}[-3\;4\;-3\;1]^{\text{T}},\;\varepsilon_2=\dfrac{\sqrt{35}}{35}[-3\;4\;-3\;1]^{\text{T}}$, $\beta_3=A_3-(A_3,\varepsilon_1)\varepsilon_1-(A_3,\varepsilon_2)\varepsilon_2=\dfrac{1}{5}[-2\;1\;3\;-1]^{\text{T}},\;\varepsilon_3=\dfrac{\sqrt{15}}{15}[-2\;1\;2\;-1]^{\text{T}}$,

QR 分解得到

$$
A=\begin{bmatrix}
    \frac{1}{\sqrt{7}}&-\dfrac{3}{\sqrt{35}}&-\dfrac{3}{\sqrt{15}}\\
    \dfrac{1}{\sqrt{7}}&\dfrac{4}{\sqrt{35}}&\dfrac{1}{\sqrt{15}}\\
    \dfrac{2}{\sqrt{7}}&\dfrac{1}{\sqrt{35}}&-\dfrac{1}{\sqrt{15}}
\end{bmatrix}\begin{bmatrix}
    \sqrt{7}&\dfrac{10\sqrt{7}}{7}&\dfrac{12\sqrt{7}}{7}\\
    &\dfrac{\sqrt{35}}{7}&-\dfrac{8\sqrt{35}}{35}\\
    &&\dfrac{\sqrt{15}}{5}
\end{bmatrix}
$$

$$
\begin{cases}
    Qy=b\\
    Rx=y\\
\end{cases},
\begin{cases}
    y=Q^{-1}b,\\
    x=R^{-1}y=R{-1}Q^{-1}b
\end{cases}
$$

解得 \mybox{$x=[-1,\;0,\;1]^{\text{T}}$}


\vspace{12pt}

40. 求矩阵 $A=\begin{bmatrix}
    2&1\\
    0&2\\
    1&0\\
\end{bmatrix}$ 的奇异值及奇异值分解.
 
\textbf{Sol}: 由于 $A^{\text{T}}A=\begin{bmatrix}
    5&2\\2&5\\
\end{bmatrix}$ 的特征值为 $\lambda_1=3,\;\lambda_2=7$, 故 $A$ 的奇异值为\mybox{ $\sqrt{3},\;\sqrt{7}$}. 其中, $A^{\text{T}}A$ 属于特征值 3 的单位特征向量为 $v_1=[\dfrac{\sqrt{2}}{2}\;-\dfrac{\sqrt{2}}{2}]^{\text{T}}$, 属于特征值 $7$ 的单位特征向量为 $v_2=[\dfrac{\sqrt{2}}{2}\;\dfrac{\sqrt{2}}{2}]^{\text{T}}$.

$V=[v_1|v_2]=\dfrac{\sqrt{2}}{2}\begin{bmatrix}
    1&1\\-1&1\\
\end{bmatrix}$

$U_1=Av_1\Sigma_1^{-1}=\begin{bmatrix}
    2&1\\
    0&2\\
    1&0\\
\end{bmatrix}\dfrac{\sqrt{2}}{2}\begin{bmatrix}
    1\\-1\\
\end{bmatrix}\begin{bmatrix}
    \sqrt{3}
\end{bmatrix}^{-1}=\dfrac{\sqrt{6}}{6}\begin{bmatrix}
    1\\-2\\1\\
\end{bmatrix}$

同理可得
$U_2=AV_2\Sigma_2^{-1}=\dfrac{\sqrt{14}}{14}\begin{bmatrix}
    3\\2\\1\\
\end{bmatrix}$

$$
U=[U_1|U_2]=\begin{bmatrix}
    \dfrac{1}{\sqrt{6}}&\dfrac{3}{\sqrt{14}}\\
    -\dfrac{2}{\sqrt{6}}&\dfrac{2}{\sqrt{14}}\\
    \dfrac{1}{\sqrt{6}}&\dfrac{1}{\sqrt{14}}\\
\end{bmatrix}
$$

可验证得到 $U^{\text{H}}AV=\begin{bmatrix}
    \sqrt{3}&\\&\sqrt{7}\\
\end{bmatrix}$

因此 

\mybox{$A=U\begin{bmatrix}
    \sqrt{3}&\\&\sqrt{7}\\
\end{bmatrix}V^T=\dfrac{\sqrt{2}}{2}\begin{bmatrix}
    \dfrac{1}{\sqrt{6}}&\dfrac{3}{\sqrt{14}}\\
    -\dfrac{2}{\sqrt{6}}&\dfrac{2}{\sqrt{14}}\\
    \dfrac{1}{\sqrt{6}}&\dfrac{1}{\sqrt{14}}\\
\end{bmatrix}\begin{bmatrix}
    \sqrt{3}&\\&\sqrt{7}\\
\end{bmatrix}\begin{bmatrix}
    1&-1\\1&1\\
\end{bmatrix}$}

\vspace{12pt}

