

31. 从两批电子元件中随机抽取一些样品, 测得它们的电阻(单位:欧姆)如下: \par 
甲批: $0.140;\;0.138;\;0.143;\;0.142;\;0.144;\;0.137;\;$  \par
乙批: $0.135;\;0.140;\;0.142;\;0.136;\;0.138;\;0.140;\;$

假定这两批电子元件的电阻都服从正态分布, 问在显著性水平 0.05 下, 能否认为这两个正态总体的方差相等?

\textbf{Sol}: 由所给数据可得 $\bar{x}=0.1407,\;\bar{y}=0.1385,\;s_1^2\approx7.867\times 10^{-6},\;s_2^2=7.1\times10^{-6}$.

得到检验假设 $H_0:\sigma_1^2=\sigma_2^2,\quad H_1:\sigma_1^2\not=\sigma_2^2$.

应用 $\text{F}$ 检验, 得到检验统计量的观测值为 $\dfrac{s_1^2}{s_2^2}=1.108$, 而在显著水平 $\alpha=0.05$ 下, $F_{0.975}(5,5)=7.1464,\;F_{0.025}(5,5)=F^{-1}_{0.975}(5,5)=0.1399$. 显然 $0.1399<1.108<7.1464$, 
所以不能拒绝 $H_0$, 
\mybox{可以认为这两个正态总体的方差相等}。


\vspace{12pt}

32. 电话交换台每分钟接到互换的次数服从 $\text{Poisson}$ 分布 $P(\lambda)$, 今观测了100个时段, 每个时段一分钟, 共有585次呼唤, 问在显著性水平0.10下, 能否认为该电环交换台每分钟接到互换的次数服从 $\lambda=6$ 的 $\text{Poisson}$ 分布?

\textbf{Sol}: 检验假设 $H_0:\lambda=6,\;H_1:\lambda\not=6$.

根据泊松分布, 检验统计量的观测值为 $\bar{x}=5.85$, 
由于 $n=100$ 较大, 所以近似地有: $\sqrt{n}\dfrac{\bar{x}-E(x)}{s}\sim N(0;1)$

由 $\text{Poisson}$ 分布性质可得, $E(x)=\lambda,\;s=\sqrt{\lambda}$, 因此检验统计量观测值为:$\sqrt{100}\dfrac{|5.85-6|}{\sqrt{6}}\approx0.6124$, 又因为 $\alpha=0.1$, 因此 $\u_{1-\frac{\alpha}{2}}=u_{0.95}\approx1.645$, 显然 $1.645>0.6124$, 所以不能拒绝 $H_0$。即可以认为\mybox{服从 $\lambda=6$ 的 $\text{Poisson}$ 分布}.




\vspace{12pt}

33. 设 $(X_1,X_2,\cdots,X_n)$ 是取自正态总体 $N(\mu;1)$ 的样本, 其中 $\mu$ 未知, 要检验假设

$$
H_0:\mu\geqslant0;\;H_1:\mu<0
$$
在显著性水平 $\alpha$ 下, 采用拒绝域为

$$
W_1=\big\{(x_1,x_2,\cdots,x_n)\big|\sqrt{n}\cdot\bar{x}<-u_{1-\alpha}\big\}
$$
的 $\mu$ 检验。

\begin{enumerate}[(1)]
    \item 求这个 $u$ 检验的功效函数 $\beta(\mu)$
    \item 当 $\alpha=0.05$时, 如果要求 $\mu\leqslant-0.1$ 时, 这个 $u$ 检验的 $\text{\uppercase\expandafter{\romannumeral2}}$ 类风险不大于 0.05, 那么样本容量 $n$ 至少应取多大?
\end{enumerate}

\textbf{Sol}: 

\begin{enumerate}[(1)]
    \item 该检验得功效函数为
    $$
    \begin{aligned}
    \beta(\mu)=&P_{\mu}\big\{(X_1,X_2,\cdots,X_n)\in W_1\big\}\\
    =&P_{\mu}\big\{\sqrt{n}\cdot\bar{X}<-\mu_{1-\alpha}\big\}\\
    =&P_{\mu}\big\{\sqrt{n}\cdot(\bar{X}-\mu)<-\mu_{1-\alpha}-\sqrt{n}\cdot\mu\big\}
    \end{aligned}
    $$
    由于 $\bar{X}\sim N(\mu;\dfrac{1}{n})$, 故 $\sqrt{n}(\bar{X}-\mu)\sim N(0;1)$, 所以记 $\Phi(x)$ 为标准正态分布 $N(0;1)$ 的分布函数, 则有
    $$
    \mybox{
        $\beta(\mu)=1-\Phi(u_{1-\alpha}+\sqrt{n}\cdot \mu),\;\mu\in\Theta=\{\mu_0;\mu_1\}$
    }.
    $$
    
    \item 依题意 $1-\beta(\mu)\leqslant 0.05$, 则 $\Phi(u_{1-\alpha}+\sqrt{n}\cdot \mu)\geqslant 1-\beta(\mu)$, 
    $\sqrt{n}\geqslant \dfrac{u_{1-\alpha}+\Phi^{-1}(1-\beta(\mu))}{\mu}=\dfrac{u_{0.95}+u_{0.95}}{0.1}\approx32.898$,$n\geqslant32.898^2\approx 1082.28$, 即样本数量至少为 \mybox{1083} 个。
\end{enumerate}

\vspace{12pt}



