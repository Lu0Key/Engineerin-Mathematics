


41. 已知 $\mathbb{R}^3$ 的两个基

$$
\mathscr{R}_1=\left\{
    \begin{bmatrix}
        1\\-1\\0\\
    \end{bmatrix},
    \begin{bmatrix}
        0\\1\\-1\\
    \end{bmatrix},
    \begin{bmatrix}
        0\\0\\1\\
    \end{bmatrix}
\right\},\quad
\mathscr{R}_2=\left\{
    \begin{bmatrix}
        1\\-1\\1\\
    \end{bmatrix},
    \begin{bmatrix}
        0\\1\\1\\
    \end{bmatrix},
    \begin{bmatrix}
        1\\0\\1\\
    \end{bmatrix}
\right\}.
$$
\begin{enumerate}[(1)]
    \item 求 $\mathscr{R}_1$ 到 $\mathscr{R}_2$ 的基变换矩阵 $\textbf{P}$;
    \item 求在 $\mathscr{R}_1$、$\mathscr{R}_2$ 下有相同坐标的所有向量.
\end{enumerate}


\textbf{Sol}:  

\begin{enumerate}[(1)]
    \item \mymathbox{
    \mathscr{R}_2=\mathscr{R}_1P,\Longrightarrow \textbf{P}= \mathscr{R}_1^{-1}\mathscr{R}_2=\begin{bmatrix}
        1&0&1\\0&1&1\\1&2&2\\
    \end{bmatrix}}
    
    \item 即求 $\mathscr{R}_1x=\mathscr{R}_2x,\Longrightarrow x= \mathscr{R}_1^{-1}\mathscr{R}_2x$, $\textbf{x}$ 的非零向量解即求矩阵 $P$ 关于特征值 1的特征向量, 若不存在特征值1则 $\textbf{x}$ 只有零向量解.  
    $$
    \begin{aligned}
        &|\lambda I-P=|=\begin{bmatrix}
        \lambda-1&0&-1\\
        0&\lambda-1&-1\\
        -1&-2&\lambda-2\\
        \end{bmatrix}\\
        =&(\lambda-1)^2
        (\lambda-2)-3(\lambda-1)\\
        =&(\lambda-1)(\lambda^2-3\lambda-1)
    \end{aligned}
    $$
    得其中一个特征值为1, 
    $$
    (P-I)x=\begin{bmatrix}
        0&0&1\\
        0&0&1\\
        1&2&1\\
    \end{bmatrix}\\
    \longrightarrow\begin{bmatrix}
        0&0&0\\0&0&1\\1&2&0
    \end{bmatrix}
    $$
    得到一个基础解系 $p=[2,-1,0]^{\text{T}}$, 
    因此在 $\mathscr{R}_1,\;\mathscr{R}_2$ 下有相同坐标的所有向量为 \mymathbox{[2,-1,0]^{\text{T}},\;k\in\mathbb{R}}.

\end{enumerate}



\vspace{12pt}

42. 已知 $\mathbb{R}^3$ 的线性变换 $\mathscr{F}$ 在基 $\mathscr{R}_1=\left\{
    \begin{bmatrix}
        -1\\1\\1\\
    \end{bmatrix},
    \begin{bmatrix}
        1\\0\\-1\\
    \end{bmatrix},
    \begin{bmatrix}
        0\\1\\1\\
    \end{bmatrix}
\right\}$ 下的矩阵是

$$
\textbf{A}=\begin{bmatrix}
    1&0&1\\
    1&1&0\\
    -1&2&1\\
\end{bmatrix},
$$

求 $\mathscr{F}$ 在基 $\mathscr{R}_2=\left\{
    \begin{bmatrix}
        1\\-1\\1\\
    \end{bmatrix},
    \begin{bmatrix}
        1\\0\\1\\
    \end{bmatrix},
    \begin{bmatrix}
        1\\1\\2\\
    \end{bmatrix}
\right\}
$ 下的矩阵 $\textbf{B}$.

\textbf{Sol}:  设 $\mathscr{R}_1$ 到 $\mathscr{R}_2$ 得基变换矩阵为 $\textbf{P}$, 则有 $\mathscr{R}_2=\mathscr{R}_1\textbf{P}$,

解得 $\textbf{P}=\mathscr{R}_1^{-1}\mathscr{R}_2=\begin{bmatrix}
    -3&-2&-2\\
    -2&-1&-1\\
    2&2&3\\
\end{bmatrix}$

可得 $\textbf{P}^{-1}=\begin{bmatrix}
    1&-2&0\\
    -4&5&-1\\
    2&-2&1\\
\end{bmatrix}$

$$
\mymathbox{
\textbf{B}=\textbf{P}^{-1}\textbf{AP}=\begin{bmatrix}
    9&6&7\\
    -22&-17&-22\\
    9&8&11\\
\end{bmatrix}}
$$


\vspace{12pt}




