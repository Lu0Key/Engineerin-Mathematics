
13. 取 $\sqrt{2}\approx  1.4$, 欲计算 $(\sqrt{2}-1)^6$ 的近似值, 有下列四个算式可采用:
\begin{enumerate}[(1)]
    \item $\dfrac{1}{(\sqrt{2}+1)^6}$;
    \item $(3-2\sqrt{2})^3$;
    \item $\dfrac{1}{(3+2\sqrt{2})^3}$;
    \item $99-70\sqrt{2}$.
\end{enumerate}
分析这四个算式哪一个所得的误差最小。

\textbf{Sol}:

$x=(\sqrt{2}-1)^6\approx 0.4142^6\approx5.0506\times 10^{-3}$
\begin{enumerate}
    \item $x\approx x_1^*=\dfrac{1}{(1.4+1)^6}=\dfrac{1}{2.4^6}\approx5\times 10^{-3}$, 误差 $e_1=x_1^*-x\approx -5.06\times 10^{-5}$;
    \item $x\approx x_2^*=(3-2\sqrt{2})^3=0.2^3\approx8\times10^{-3}$, 误差 $e_2=x_2^*-x\approx 2.9494\times 10^{-3}$;
    \item $x\approx x_3^*=\dfrac{1}{(3+2\sqrt{2})^3}=\dfrac{1}{5.8^3}\approx 5\times 10^{-3}$, 误差 $e_3=x_3^*-x\approx -5.06\times10^{-5}$;
    \item $x\approx x_4^*=99-70\sqrt{2}\approx 1.0$, 误差 $e_4=x_4^*-x=\approx 9.95\time 10^{-1}$.
\end{enumerate}
我们发现,\mybox{第 (1) 和第 (3) 个算式的误差最小}.

\vspace{12pt}

14. 如何计算下列函数值才比较准确:
\begin{enumerate}
    \item $\dfrac{1}{1+2x}-\dfrac{1-x}{1+x},\;|x|\ll 1$;
    \item $\sqrt{x+\dfrac{1}{x}}-\sqrt{x-\dfrac{1}{x}},|x|\gg 1$;
    \item $\dfrac{1-\cos x}{x},\;|x|\ll 1$;
    \item $\arctan(x+1)-\arctan (x),\;|x|\gg 1$.
\end{enumerate}

\textbf{Sol}:
\begin{enumerate}
    \item 两个相近的数相减会造成误差. $\dfrac{1}{1+2x}-\dfrac{1-x}{1+x}=\mybox{$\dfrac{2x^2}{2x^2+3x+1}$}$
    \item 两个相近的数相减会造成误差. $\sqrt{x+\dfrac{1}{x}}-\sqrt{x-\dfrac{1}{x}}=\mybox{$\dfrac{2}{\sqrt{x}\Big(\sqrt{x^2+1}+\sqrt{x^2-1}\Big)}$}$
    \item 两个相近的数相减会造成误差. $\dfrac{1-\cos x}{x}=\dfrac{1-(1-\frac{x^2}{2}+\dfrac{x^4}{4!}+o(x^4))}{x}\approx\mybox{$\dfrac{x}{2}-\dfrac{x^3}{24}$}$
    \item 两个相近的数相初造成误差. $\arctan(x+1)-\arctan(x)=\arctan\dfrac{1}{x^2+x+1}\approx\dfrac{1}{x^2+x+1}-\dfrac{1}{3(x^2+x+1)^3}\approx\mybox{$\dfrac{1}{x^2}$}$
\end{enumerate}


\vspace{12pt}

15. 设总体 $X$ 服从正态分布 $N(12,2^2)$, 现在随机抽取容量为 5 的样本, 问:
\begin{enumerate}
    \item 此样本最小值小于10的概率是多少?
    \item 此样本最大值大于15的概率是多少?
\end{enumerate}

\textbf{Sol}: 易知 $\dfrac{X-\mu}{\sigma}=\dfrac{X-12}{2}$ 服从标准正态分布,
\begin{enumerate}
    \item 设 $X_1,X_2,X_3,X_4,X_5$ 为样本, 
    则 $p=Pr(\min X_i<10)=1-Pr(\min X_i\geqslant 10)=1-\prod_{i=1}^5Pr(X_i\geqslant10)=1-[Pr(X_i\geqslant10)]^5$,
    $Pr(X_i\geqslant10)=Pr\big(\dfrac{X_i-\mu}{\sigma}\geqslant\dfrac{10-\mu}{\sigma}\big)=1-\Phi(\dfrac{10-12}{2})=1-\Phi(-1)\approx\mybox{$0.8413$}$. 
    $p=1-[Pr(X_i\geqslant10)]^5\approx 1- 0.8413^5\approx 0.5785$
    \item 同理, 设 $X_1,X_2,X_3,X_4,X_5$ 为样本, 则
    $q=Pr(\max_i\leqslant 15)=1-[Pr(X_i\leqslant 15)]^5=1-\Phi^5(\dfrac{15-12}{2})\approx\mybox{0.2923}$.
\end{enumerate}



