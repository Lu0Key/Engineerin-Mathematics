
34.验证 $A=\begin{bmatrix}
    -\sqrt{2}i&-4\\
    4&\sqrt{2}i
\end{bmatrix}$ 是正规矩阵, 并求酉矩阵 $U$ 使得 $U^{-1}AU$ 为对角矩阵.


\textbf{Sol}:  因为 $A^HA=\begin{bmatrix}
    18&\\&18
\end{bmatrix}=AA^H$, 
\mytextbox{所以 $A$ 是正规矩阵}.

根据 $\text{Schur}$ 定理, 存在酉矩阵 $U$ 使 $U^{-1}AU$ 为对角矩阵。 由 $|A-\lambda I|=0$ 算出特征值 $\lambda_1=3\sqrt{2}i,\;\lambda_2=-3\sqrt{2}i$. 对应特征值 $3\sqrt{2}i$ 的特征向量 $\varepsilon_1=\dfrac{1}{\sqrt{3}}\begin{bmatrix}
    i\\\sqrt{2}
\end{bmatrix}$, 
以 $\varepsilon_1$ 为第一列的一个酉矩阵可取为 
\mymathbox{
U=\dfrac{1}{\sqrt{3}}\begin{bmatrix}
    i&-\sqrt{2}\\\sqrt{2}&-i\\
\end{bmatrix}}

可以验证 $U^{-1}AU=\begin{bmatrix}
    3\sqrt{2}i&\\&-3\sqrt{2}i
\end{bmatrix}$.


\vspace{12pt}

35.求 $A=\begin{bmatrix}
    1+i&3\\
    2&1-i
\end{bmatrix}$ 的谱半径 $\rho(A)$.


\textbf{Sol}: 由 $|A-\lambda I|=0$ 算出特征值 $\lambda_1=1+\sqrt{5},\;\lambda_2=1-\sqrt{5}$. 

于是 $S_p(A)=\{1-\sqrt{5},\;1+\sqrt{5}\}$, 所以 \mymathbox{\rho(A)=1+\sqrt{5}}.


\vspace{12pt}

36.用 $\text{Gauss}$ 消去法和列主元消去法求解

$$
\begin{cases}
    4x_1-1.24x_2+0.3x_3=-11.04,\\
    2x_1+4.5x_2+0.36x_3=0.02,\\
    0.5x_1+1.1x_2+3.1x_3=6.\\
\end{cases}
$$


\textbf{Sol}: 根据题意, 方程组的增广矩阵为

$$
\begin{bmatrix}
    4&-1.24&0.3&-11.04\\
    2&4.5&0.36&0.02\\
    0.5&1.1&3.1&6\\
\end{bmatrix}
$$

通过高斯消去法得

$$
\begin{aligned}
    &\begin{bmatrix}
        \mymathbox{4}&-1.24&0.3&-11.04\\
        2&4.5&0.36&0.02\\
        0.5&1.1&3.1&6\\
    \end{bmatrix}\\
    \Longrightarrow&
    \begin{bmatrix}
        4&-1.24&0.3&-11.04\\
        0&\mymathbox{5.12}&0.21&5.54\\
        0&1.26&3.062&7.38\\
    \end{bmatrix}\\
    \Longrightarrow&
    \begin{bmatrix}
        4&-1.24&0.3&-11.04\\
        0&5.12&0.21&5.54\\
        0&0&3.001&6.017
    \end{bmatrix}
\end{aligned}
$$

回代求解得到

$$
\mymathbox{
\begin{aligned}
    &x_3=\dfrac{6.017}{3.011}=2\\
    &x_2=\dfrac{5.54-0.21x_3}{5.12}=1\\
    &x_1=\dfrac{-11.04-0.3x_3+1.24x_2}{4}=-2.6
\end{aligned}}
$$

\vspace{12pt}


