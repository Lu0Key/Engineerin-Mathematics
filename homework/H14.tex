
48. 给定数据表

\begin{center}
    \begin{tabular}{c|ccc}
        \hline
        $x_i$ & 1.2 & 3.2 & 4.5\\
        $f_i$ & 101 & 112 & 109\\
        \hline
    \end{tabular}
\end{center}

求通过这三个点的次数不超过 2 的插值多项式.

\textbf{Sol}:  

$$
\begin{bmatrix}
    1&1.2&1.44\\
    1&3.2&10.24\\
    1&4.5&20.25\\
\end{bmatrix}\begin{bmatrix}
    a_0\\a_1\\a_2\\
\end{bmatrix}=\begin{bmatrix}
    101\\112\\109\\
\end{bmatrix}\Rightarrow\begin{bmatrix}
    a_0\\a_1\\a_2\\
\end{bmatrix}=\begin{bmatrix}
    85.31\\15.91\\-2.366\\
\end{bmatrix}
$$

解得 \mybox{$y=85.31+15.91x-2.366x^2$}.

\vspace{12pt}


49. 已知数据表

\begin{center}
    \begin{tabular}{c|ccccc}
        \hline
        $x$ & 0.1 & 0.2 & 0.3 & 0.4 & 0.5\\
        $y(x)$ & 0.70010 & 0.40160 & 0.10810 &-0.17440 & -0.43750\\
        \hline
    \end{tabular}
\end{center}

用反插值(即在 $y=y(x)$ 的反函数 $x=x(y)$ 存在的假设下, 构造反函数 $x=x(y)$ 的插值多项式) 
求 $y(x) = 0$ 在 $(0.3, 0.4)$ 内的根的近似值.


\textbf{Sol}:  

差商表如下:

% Please add the following required packages to your document preamble:
% \usepackage[normalem]{ulem}
% \useunder{\uline}{\ul}{}
\begin{center}
\begin{tabular}{c|c|c|c|c|c}
\hline
x(y)    & x   & 一阶差商          & 二阶差商          & 三阶差商           & 四阶差商          \\ \hline
0.7001  & 0.1 &               &               &                &               \\
0.4016  & 0.2 & {\underline{-0.3350}} &               &                &               \\
0.1081  & 0.3 & -0.3407       & {\underline{0.00964}} &                &               \\
-0.1744 & 0.4 & -0.3540       & 0.02303       & {\underline{-0.01531}} &               \\
0.4375  & 0.5 & -0.3801       & 0.04784       & -0.02956       & {\underline{0.01253}}
\end{tabular}
\end{center}

根据牛顿插值多项式

$x=x(y)=f(y_0)+f[y_0,y_1](y-y_0)+f[y_0,y_1,y_2](y-y_0)(y-y_1)+f[y_0,y_1,y_2,y_3](y-y_0)(y-y_1)(y-y_2)+f[y_0,y_1,y_2,y_3,y_4](y-y_0)(y-y_1)(y-y_2)(y-y_3)$,

因此 $y(x)=0$ 的根的近似值为 $x=x(0)=0.1-0.3350(-0.7001)+0.00964(-0.7001)(-0.4016)-0.01531(-0.7001)(-0.4016)(-0.1081)+0.01253(-0.7001)(-0.4016)(-0.1081)(0.1744)=\mybox{0.3376}$

\vspace{12pt}


50. 已知实验数据表

\begin{center}
    \begin{tabular}{c|ccccc}
        \hline
        $x_i$ & 19 & 25 & 31 & 38 & 44\\
        $y_i$ & 19.0 & 32.3 & 49.0 & 73.3 & 97.8\\
        \hline
    \end{tabular}
\end{center}

用最小二乘法求形如 $y=a+bx^2$ 的经验公式, 并计算均方误差.

\textbf{Sol}:  

以 $\varphi_0(x)=1,\;\varphi_1(x)=x^2$ 为基函数, 且

$y=[19.0\;32.3\;49.0\;73.3\;97.8]^{\text{T}}$, $a=[a\;b]^{\text{T}}$, $G=\begin{bmatrix}
    1&1&1&1&1\\
    361&625&961&1444&1936\\
\end{bmatrix}^{\text{T}}$, $G^{\text{T}}G=\begin{bmatrix}
    5&5327\\
    5327&7277699\\
\end{bmatrix}$

$G^{\text{T}}y=\begin{bmatrix}
    271.4\\369321.5\\
\end{bmatrix}$

$$
\begin{bmatrix}
    5&5327\\
    5327&7277699\\
\end{bmatrix}\begin{bmatrix}
    a\\b\\
\end{bmatrix}=\begin{bmatrix}
    271.4\\369321.5\\
\end{bmatrix}
$$

\vspace{120pt}

解得 $\begin{bmatrix}
    a\\b\\
\end{bmatrix}=\begin{bmatrix}
    0.9726\\0.05004\\
\end{bmatrix}$

因此\mybox{$y=0.9726+0.5004x^2$}, 可计算得误差平方和为 $\sigma^2=\dfrac{1}{5}\sum(y-\hat{y})^2=0.003039$.

\vspace{12pt}



