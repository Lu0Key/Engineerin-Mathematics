
7. 求可逆矩阵 $P$, 使得 $PA$ 为 $\text{Hermite}$ 阶梯型矩阵, 其中: 
$$
A=\begin{bmatrix}
1&1&2&2&3\\
2&2&3&1&4\\
1&0&1&1&5\\
2&3&5&5&4\\
\end{bmatrix}
$$

\textbf{Sol}:
由于 $P[A:I]=[PA:P]=[H:P]$, 构造矩阵
$$
\begin{aligned}
&\left[\begin{array}{ccccc|cccc}
    1&1&2&2&3&1&0&0&0\\
    2&2&3&1&4&0&1&0&0\\
    1&0&1&1&5&0&0&1&0\\
    2&3&5&5&4&0&0&0&1\\
\end{array}\right]\\
\Rightarrow&
\left[\begin{array}{ccccc|cccc}
    1&0&0&-2&3&-2&1&1&0\\
    0&1&0&-2&-4&-1&1&-1&0\\
    0&0&1&3&2&2&-1&0&0\\
    0&0&0&0&0&-3&0&1&1\\
\end{array}\right]\\
\end{aligned}
$$

因此得到 
$$
\mybox{
$P=\left[\begin{array}{cccc}
    -2&1&1&0\\
    -1&1&-1&0\\
    2&-1&0&0\\
    -3&0&1&1\\
\end{array}\right]$
}
$$

\vspace{12pt}

8. 求矩阵 $A$ 的一个满秩分解, 其中:
$$
A=\begin{bmatrix}
    1&-1&2&0&-1\\
    1&1&-2&2&3\\
    3&-1&8&5&1\\
    1&3&-6&4&7\\
\end{bmatrix}
$$

\textbf{Sol}:

对 $A$ 进行初等行变换得到
$$
\begin{aligned}
    A=&\begin{bmatrix}
    1&-1&2&0&-1\\
    1&1&-2&2&3\\
    3&-1&8&5&1\\
    1&3&-6&4&7\\
    \end{bmatrix}\\
    \Rightarrow&
    \begin{bmatrix}
        1&0&0&1&1\\
        0&1&0&2&2\\
        0&0&1&\frac{1}{2}&0\\
        0&0&0&0&0
    \end{bmatrix}
\end{aligned}
$$

因此将 $A$ 分解为 $A=F\cdot G$, 其中

$$
\mybox{
$F=\begin{bmatrix}
    1&-1&2\\1&1&-2\\3&-1&-6\\1&3&-6
\end{bmatrix}$
},
\mybox{
$G=\begin{bmatrix}
    1&0&0&1&1\\0&1&0&2&2\\0&0&1&\frac{1}{2}&0
\end{bmatrix}$
}
$$

\vspace{12pt}

9. 设 $x$ 的相对误差为 $1\%$, 求 $x^n$ 的相对误差.

\textbf{Sol}:

$e_r=\dfrac{e}{x^*}=\dfrac{x^*-x}{x^*}=1\%$,\\
$e_r'=\dfrac{e'}{(x^*)^n}=\dfrac{(x^*)^n-x^n}{(x^*)^n}=1-\left(\dfrac{x}{x^*}\right)^n=\mybox{$1-0.99^n$}$.

\vspace{12pt}

10. 从总体 $N(20;(\sqrt{3})^2)$ 中抽取容量分别为 10 和 15 的两个相互独立的样本, 求这两个样本均值之差的绝对值小于 0.3 的概率. 这里 $\displaystyle \bar{x}=\dfrac{1}{n}\sum_{i=1}^n x_i$, $\displaystyle \bar{y} = \dfrac{1}{n}\sum_{i=1}^n y_i$, $\displaystyle s_x^2=\dfrac{1}{n-1}(x_i-\bar{x})^2$, $\displaystyle s_y^2=\dfrac{1}{n-1}\sum_{i=1}^n(y_i-\bar{y})^2$.

\textbf{Sol}:

易知 $Y_{10}\sim N\bigg(20,\dfrac{3}{10}\bigg)$, $Y_{15}\sim N\bigg(20,\dfrac{1}{5}\bigg)$, 则 $Y_{10}-Y_{15}\sim N\bigg(0,\dfrac{1}{2}\bigg)$, $Pr(|Y_{10}-Y_{15}|<0.3)=2\cdot \Phi(0.3\sqrt{2})-1=2\cdot\Phi(0.4243)-1$, 查表可得 $\mybox{$Pr(|Y_{10}-Y_{15}|<0.3)=0.3286$}$.

\vspace{12pt}

11. 设 $X_1,\cdots,X_5$ 是取自总体 $N(0;1)$ 的样本,
\begin{enumerate}[(1)]
    \item 求常数 $c_1,d_1$, 使 $c_1(X_1+X_2)^2+d_1(X_3+X_4+X_5)^2$ 服从 $\chi^2$ 分布, 并指出其自由度.
    \item 求常数 $c_2,d_2$, 使 $\dfrac{c_2(X_1^2+X_2^2)}{d_2(X_3+X_4+X_5)^2}$ 服从 $\text{F}$ 分布, 并指出其自由度.
\end{enumerate}

\textbf{Sol}:

(1) 已知 $\dfrac{X_1+X_2}{\sqrt{2}}\sim N(0,1)$, $\dfrac{X_3+X_4+X_5}{\sqrt{3}}\sim N(0,1)$, 因此有 $\Bigg(\dfrac{X_1+X_2}{\sqrt{2}}\Bigg)^2\sim \chi^2(1)$, $\Bigg(\dfrac{X_3+X_4+X_5}{\sqrt{3}}\Bigg)^2\sim\chi^2(1)$, $\Bigg(\dfrac{X_1+X_2}{\sqrt{2}}\Bigg)^2+\Bigg(\dfrac{X_3+X_4+X_5}{\sqrt{3}}\Bigg)^2\sim\chi^2(2)$. \\
因此得 \mybox{$c_1=\dfrac{1}{2},\;d_1=\dfrac{1}{3}$}. 自由度为 2.

(2) 由上可知 $X_1^2+X_2^2\sim \chi^2(2)$, $\Bigg(\dfrac{X_3+X_4+X_5}{\sqrt{3}}\Bigg)^2\sim \chi^2(1)$, 因此有 $\dfrac{\frac{X_1^2+X_2^2}{2}}{\frac{(x_3+X_4+X_5)^2}{3}}\sim F(2,1)$,\\
因此得 \mybox{$c_2=3,\;d_2=2$}, 第一自由度为2, 第二自由度为1.

\vspace{12pt}

12. $\lambda$ 取何值时, 线性方程组

$$
\begin{cases}
(2-\lambda)x_1+2x_2-2x_3=1,\\
2x_1+(5-\lambda)x_2-4x_3=2,\\
-2x_1-4x_2+(5-\lambda)x_3=-\lambda-1\\
\end{cases}
$$
有唯一解、无解或无穷多解?在有无穷多解时, 求其一般解.

\textbf{Sol}:

构造增广矩阵 $B=[A|b]$, 再进行初等行变换, 其中 $A$ 为系数矩阵.

$$
\begin{aligned}
    &\left[\begin{array}{ccc|c}
        2-\lambda&2&-2&1\\
        2&5-\lambda&-4&2\\
        -2&-4&5-\lambda&-\lambda-1
    \end{array}\right]\\
    \Rightarrow&
    \left[\begin{array}{ccc|c}
        2-\lambda&5-\lambda&-4&2\\
        0&2-\frac{(5-\lambda)(2-\lambda)}{2}&2-2\lambda&\lambda-1\\
        0&1-\lambda&1-\lambda&1-\lambda
    \end{array}\right]\\
\end{aligned}
$$

i) \mybox{当 $\lambda=1$ 时, 方程组有无穷多解.}  

$$
B\Rightarrow\begin{bmatrix}
    1&2&-2&1\\0&0&0&0\\0&0&0&0
\end{bmatrix}
$$

此时一般解为

$$
\mybox{
$\begin{cases}
    x_1=k_1\\
    x_2=k_2\\
    x_3=\frac{k_1+2k_2-1}{2}\\
\end{cases}$
}
$$

ii) 当 $\lambda\not =0$ 时, 有
$$
B\Rightarrow\begin{bmatrix}
    2&5-\lambda&-4&2\\
    0&0&\frac{(5-\lambda)(2-\lambda)}{2}-2\lambda&\lambda+\frac{(5-\lambda)(2-\lambda)}{2}-3\\
    0&1&1&1\\
\end{bmatrix}
$$

当 $\lambda=10$ 时, 有 
$B\Rightarrow
\begin{bmatrix}
    2&-5&-4&2\\0&0&0&1\\0&1&1&1\\
\end{bmatrix}$,
$r(A)<r(B)$, 因此, \\
\mybox{当 $\lambda=10$ 时 此时方程组无解}.

iii) 当 $\lambda\not=1$ 且 $\lambda\not=10$ 时, $r(A)=r(B)$, 因此, \mybox{当 $\lambda\not=1$ 且 $\lambda\not=10$ 时, 方程有唯一解}.

\vspace{12pt}