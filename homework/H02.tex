

6. 设 $X_1,X_2,X_3$ 是取自正态总体 $X$ 的样本, 就下列总体 $X$ 给出这个样本的概率函数或概率密度:
\begin{enumerate}[(1)]
    \item $X$ 服从参数为 $\lambda$ 的 $\text{Poission}$ 分布;
    \item $X$ 服从参数为 $\lambda$ 的 指数分布, 即概率密度为 $f(x)=\begin{cases}\lambda e^{-\lambda x},&x>0,\\0,&x\leqslant0.\end{cases}$;
    \item $X$ 在区间 $(a,b)$ 上服从均匀分布;
    \item  $X$ 服从参数为 $a,b$ 的 $\beta$ 分布, 即概率密度为 $f(x;a,b)=\begin{cases}
        \dfrac{\Gamma(a+b)}{\Gamma(a)\Gamma(b)}x^{a-1}(1-x)^{b-1},&0<x<1,\\0,&\text{其他}.
    \end{cases}$, 记 $X\sim B(a,b)$.
\end{enumerate}


\textbf{Sol}: 

(1) $X$ 服从参数为 $\lambda$ 的 $Poisson$ 分布为 $P(X=k)=\dfrac{\lambda^k}{k!}e^{-\lambda},k=0,1,\cdots$, 
$P(X_1=x_1,X_2=x_2,X_3=x_3)=\mymathbox{\dfrac{\lambda_{x_1+x_2+x_3}\cdot e^{-3\lambda}}{x_1!\cdot x_2!\cdot x_3!}}$.\par

(2) $X$ 服从参数为 $\lambda$ 的 $f(x)=\begin{cases}
    \lambda e^{-\lambda x}, &x>0,\\
    0,&x\leqslant 0.
\end{cases},$  
\newline
$f(x_1,x_2,x_3)=f(x_1)f(x_2)f(x_3)=
\mymathbox{
\begin{cases}
    \lambda^3 e^{-\lambda(x_1+x_2+x_3)},&x_1>0,x_2>0,and \;x_3>0,\\
    0,&x_1\leqslant 0,x_2\leqslant 0,or\;x_3\leqslant0.
\end{cases}}$\par


(3) $X$ 在区间 $(a,b)$ 上服从均匀分布 $f(x)=\begin{cases}
    \frac{1}{b-a},&a<x<b\\
    0,&else
\end{cases}$, 则 $f(x_1,x_2,x_3)=f(x_1)f(x_2)f(x_3)=
\mymathbox{
\begin{cases}
    \dfrac{1}{(b-a)^3},&a<x_1,x_2,x_3<b\\
    0,&else.
\end{cases}}$\par

(4) $X$ 服从参数为 $a,b$ 的 $\beta$ 分布, 
$$
\begin{aligned}
    &f(x_1,x_2,x_3)
    =f(x_1)f(x_2)f(x_3)\\
    =&\mymathbox{
    \begin{cases}
        \left[\dfrac{\Gamma(a+b)}{\Gamma(a)\Gamma(b)}\right]^3(x_1x_2x_3)^{a-1}\times\\
        \quad\quad\quad [(1-x_1)(1-x_2)(1-x_3)]^{b-1]},&0<x_1,x_2,x_3<1,\\
        0,&else.\\
    \end{cases}}
\end{aligned}
$$

\vspace{12pt}

