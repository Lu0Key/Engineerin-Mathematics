
1. 已知向量 $\alpha_1=[3,1,5,2]^\text{T},\;\alpha_2=[10,5,1,10]^\text{T},\;\alpha_3=[1,-1,1,4]^\text{T}$. 若有方程 $3(\alpha_1-\beta)+2(\alpha_2-\beta)=5(\alpha_3+\beta)$, 求 $\beta$.

\textbf{Sol}: 

由题 $3(\alpha_1-\beta)+2(2\alpha_2-\beta)=5(\alpha_3+\beta)$ 
可得 $10\beta=3\alpha_1+2\alpha_2-5\alpha_3$, 
因此 $\beta=\frac{3}{10}\alpha_1+\frac{2}{10}\alpha_2-\frac{5}{10}\alpha_3$, 
代入 $\alpha_1,\;\alpha_2,\;\alpha_3$ 

可得 \mymathbox{\beta=[\frac{12}{5},\frac{9}{5},\frac{6}{5},\frac{3}{5}]^\text{T}}.

\vspace{12pt}

2. 证明:$\Big\{\alpha_1=[1,1,0]^\text{T},\;\alpha_2=[0,0,2]^\text{T},\;\alpha_3=[0,1,2]^\text{T}\Big\}$ 是 $\mathbb{R}^3$ 的一个基; 并求 $\beta=[5,7,-2]^\text{T}$ 在这个基下的坐标.

\textbf{Sol}: 设 $k_1\alpha_1+k_2\alpha_2+k_3\alpha_3=0$, 则有:

$$
\begin{cases}
k_1=0\\
k_1+k_3=0\\
2k_2+2k_3=0\\
\end{cases}
\Rightarrow 
\begin{cases}
    k_1=0\\
    k_2=0\\
    k_3=0\\
\end{cases}
$$

因此 $\alpha_1,\;\alpha_2,\;\alpha_3$ 线性无关, 因此是 $\mathbb{R}^3$ 的一个基. 设 $\beta=n_1\alpha_1+n_2\alpha_2+n_3\alpha_3$。因此有

$$
\begin{cases}
    n_1=5\\
    n_1+n_3=7\\
    2n_2+2n_3=-2\\
\end{cases}
\Rightarrow
\begin{cases}
    n_1=5\\
    n_2=-3\\
    n_3=2\\
\end{cases}
$$


因此 $\beta$ 在这个基下的坐标为 \mymathbox{[5,-3,2]^\text{T}}.

\vspace{12pt}

3. 已知 $\Big\{\alpha_1=[1,1,0,0]^\text{T},\;\alpha_2=[0,0,1,1]^\text{T},\;\alpha_3=[1,0,0,-1]^\text{T},\;\alpha_4=[0,1,1,0]^\text{T}\Big\}$ 是 $\mathbb{R}^4$ 的一个基; 用 $\text{Schmidt}$ 正交化方法求 $\mathbb{R}^4$ 的标准正交基。

\textbf{Sol}: \text{Schmidt} 正交化:  
令 $\beta=\alpha_1=[1,1,0,0]^\text{T}$, $\beta_2=\alpha_2-\dfrac{(\alpha_2,\beta_1)}{(\beta_1,\beta_1)}\beta_1=[0,0,1,1]^{\text{T}}$, $\beta_3=\alpha_3-\dfrac{(\alpha_3,\beta_1)}{(\beta_1,\beta_1)}\beta_1-\dfrac{(\alpha_3,\beta_2)}{(\beta_2,\beta_2)}\beta_2=[\frac{1}{2},-\frac{1}{2},\frac{1}{2},-\frac{1}{2}]^\text{T}$.\par


再将向量单位化得 \par
$\eta_1=\dfrac{\beta_1}{|\beta_1|}=[\frac{\sqrt{2}}{2},\frac{\sqrt{2}}{2},0,0]^\text{T}$,\par
$\eta_2=\dfrac{\beta_2}{|\beta_2|}=[0,0,\frac{\sqrt{2}}{2},\frac{\sqrt{2}}{2}]^\text{T}$,\par
$\eta_3=\dfrac{\beta_3}{|\beta_3|}=[\frac{1}{2},-\frac{1}{2},\frac{1}{2},-\frac{1}{2}]^\text{T}$,\par
$\eta_4=\dfrac{\beta_4}{|\beta_4|}=[-\frac{1}{2},\frac{1}{2},\frac{1}{2},-\frac{1}{2}]^\text{T}$.\par

至此我们得到了所求的 $\mathbb{R}^4$ 的标准正交基.

\vspace{12pt}

4. 已知 $AP=PB$, 求 $A$ 和 $A^8$。其中:
$$
B=\begin{bmatrix}
1&0&0\\
0&1&0\\
0&0&-1
\end{bmatrix},\;
P=\begin{bmatrix}
    1&0&1\\
    -1&1&1\\
    0&-1&1\\
\end{bmatrix}.
$$

\textbf{Sol}: 因为有 $|P|=\begin{vmatrix}
    1&0&1\\
    -1&1&1\\
    0&-1&1\\
\end{vmatrix}=3\not=0$, 因此 $P$ 可逆, 又 $AP=PB$, 因此存在 $P^{-1}$, 使得 $A=PBP^{-1}$ 成立。

$$
\begin{aligned}
P^{-1}=&\dfrac{1}{|P|}P^{*}=\dfrac{1}{3}
\begin{pmatrix}
2&-1&-1\\
1&1&-2\\
1&1&1\\
\end{pmatrix}\\
=&\begin{pmatrix}
\frac{2}{3}&-\frac{1}{3}&-\frac{1}{3}\\
\frac{1}{3}&\frac{1}{3}&-\frac{2}{3}\\
\frac{1}{3}&\frac{1}{3}&\frac{1}{3}\\
\end{pmatrix}
\end{aligned}
$$

因此有 $\mymathbox{
A=PBP^{-1}=\begin{pmatrix}
    \frac{1}{3}&-\frac{2}{3}&-\frac{2}{3}\\
    -\frac{2}{3}&\frac{1}{3}&-\frac{2}{3}\\
    -\frac{2}{3}&-\frac{2}{3}&\frac{1}{3}\\
\end{pmatrix}}$
\par

$A^8=(PBP^{-1})\cdots(PBP^{-1})=PBP^{-1}PBP^{-1}\cdots PBP^{-1}=PB^8P^{-1}$, 又 $B^8=\begin{pmatrix}
    1&0&0\\
    0&1&0\\
    0&0&1\\
\end{pmatrix}$, 因此\\\par

$$
\mymathbox{
\begin{aligned}
    A^8=&PB^8P^{-1}=PP^{-1}\\
    =&E=\begin{pmatrix}
        1&0&0\\
        0&1&0\\
        0&0&1\\
    \end{pmatrix}
\end{aligned}}
$$

\vspace{12pt}

5. 设 $A=\begin{bmatrix}
    1&2&0&0\\
    2&6&0&0\\
    0&0&3&5\\
    0&0&1&2
\end{bmatrix}$, 求 $|A|$ 和 $|A^{-1}|$.

\textbf{Sol}: 

$\mymathbox{
    |A|=\begin{vmatrix}
    1&2\\
    2&6\\
\end{vmatrix}\cdot\begin{vmatrix}
    3&5\\
    1&2\\
\end{vmatrix}=2}\not=0$, 
因此 $A$ 可逆, 
有 $AA^{-1}=E\Rightarrow|A||A^{-1}|=1\Rightarrow \mymathbox{|A^{-1}|=\dfrac{1}{2}}$.

\vspace{12pt}

